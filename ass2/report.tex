\documentclass[12pt]{report}

\usepackage{float}
\restylefloat{table}
\restylefloat{figure}

\usepackage[margin=1.5in]{geometry}
\usepackage{graphicx}

\usepackage{setspace}
\doublespacing

\usepackage{titlesec}
\titleformat{\chapter}[display]
    {\normalfont\huge\bfseries}{\chaptertitlename\ \thechapter}{0pt}{\Huge}
\titlespacing*{\chapter}{0pt}{0pt}{20pt}

\bibliographystyle{apalike}

\makeatletter
\renewcommand*\l@chapter[2]{
    \ifnum \c@tocdepth >\m@ne
    \addpenalty{-\@highpenalty}
    \addvspace{1.0em \@plus\p@}
    \setlength\@tempdima{1.5em}
    \begingroup
    \parindent \z@ \rightskip \@pnumwidth
    \parfillskip -\@pnumwidth
    \leavevmode
    \advance\leftskip\@tempdima
    \hskip -\leftskip
    #1\nobreak
    \xleaders\hbox{$\m@th
    \mkern \@dotsep mu\hbox{.}\mkern \@dotsep mu$}
    \hfil\nobreak\hb@xt@\@pnumwidth{\hss #2}\par
    \penalty\@highpenalty
    \endgroup
    \fi
}
\makeatother

\begin{document}

\begin{singlespacing}
    \begin{titlepage}
        \begin{center}
            \textbf{UNIVERSITY OF WATERLOO}\\
            Faculty of Mathematics\\
            \vspace{2.0in}
            \textbf{A CRITICAL ANALYSIS OF LEARNING STYLES}\\
            \vspace{2.0in}
            PD2: Critical Reflection and Report Writing\\
            Waterloo, Ontario\\
            \vspace{2.0in}
            Prepared by\\
            Christopher Dueck\\
            2A Honours Mathematics\\
            ID 20411985\\
            April 2, 2012\\
        \end{center}
    \end{titlepage}
\end{singlespacing}

\pagenumbering{roman}

\tableofcontents

%% keep lof and lot on same page
\begingroup
\addcontentsline{toc}{chapter}{List of Figures}
\listoffigures

\let\clearpage\relax
\addcontentsline{toc}{chapter}{List of Tables}
\listoftables
\endgroup

\newpage

\addcontentsline{toc}{chapter}{Summary} %% 1 page

\paragraph This report, titled A Critical Analysis of Learning Styles, provides
a comprehensive overview of some of the most popular models of the theory of
learning styles, as well as a critical analysis as to the suitability of such
methods in a modern educational setting. A general explanation of learning
styles and their proposed benefits will be examined along with an thorough
investigation of the most research papers for each model.

\paragraph Although much effort has been devoted by the scientific community to
ground learning styles in scientific rigor, this report finds that there is an
alarming lack of credible scientific results to correlate the claims put forth
by proponents of learning styles. This report will advise against the direct
application of learning style methods in an educational setting and instead
provide alternative methodology using some of the more tangible results of
learning style theory.

\paragraph While this report attempts to reach definitive, demonstrable
conclusions, it acknowledges the fact that the field of human learning is highly
complex and varied. It is impossible to provide a set of general recommendations
that will benefit everyone.

\chapter*{Introduction} %% 1 page

\addcontentsline{toc}{chapter}{Introduction}
\setcounter{chapter}{1}
\pagenumbering{arabic}

\setlength{\emergencystretch}{3em}

\paragraph Since its recognition in the 1970's, the concept of learning styles
has had a widespread influence on modern education theory, although recently the
credibility of the theory has come under serious question. This report will
first cover the most popular flavours of the learning styles such as Neil
Fleming's VARK \cite{fleming2001teaching}, David Kolb's Experiential Learning
\cite{kolb1984experiential}, and Peter Honey and Alan Mumford's Learning Styles
Questionnaire \cite{honey1989learning}.  The core ideas of learning styles will
be explored, including the proposed benefits such as increased content
retention, and enhanced scores on testable material.

\setlength{\emergencystretch}{0em}

\paragraph After discussing the most prevalent models of learning styles, this
report will then present the modern critique of the theory. Although learning
styles have existed for a substantial period of time, there is an alarming lack
of scientifically rigorous study confirming the claims touted by proponents of
the theory. Though there have been attempts to ground the theory in factual,
reproducible experimentation, none have provided a satisfyingly conclusive
answer to the question of whether the application of learning style based theory
in an educational setting provides measurable benefit to the learning process.
This report will draw its own conclusions as to the suitability of learning
styles as an educational tool, and provide recommendations for any readers
interested in employing a model of learning styles in an educational setting.

%% BODY 4-6 pages

\chapter*{An Overview of Learning Styles}
\addcontentsline{toc}{chapter}{An Overview of Learning Styles}
\setcounter{chapter}{2}

\paragraph Although many distinct models of learning styles exist, they share a
core theme. Learning styles are broadly described as ``cognitive, affective, and
physiological traits that are relatively stable indicators of how learners
perceive, interact with, and respond to the learning environment''
\cite{keefe1990developing}. This suggests that an individual will be able to
learn at an increased rate, or with higher content retention, if they are taught
in a way that aligns with an innate personal preference for consuming
information.

\section {VARK Model}

\paragraph One well known model is Neil Fleming's \textit{Visual, Auditory,
Reading/Writing, Kinesthetic} or VARK model \cite{fleming2001teaching}.  Fleming
himself describes VARK as "a starting place for a conversation among teachers
and learners about learning" \cite{fleming2006learning}. VARK attempts to
classify an individual learner into the sensory category from which they benefit
the most. Although it is possible for a learner to be competent using more than
one sensory aspect, Fleming claims that information can be consumed most
efficiently when the dominant learning style of a student is thoroughly
utilized.

\begin{figure}[H]
    \centering
    \includegraphics[width=5in]{figures_tables/vark}
    \caption{The VARK model}
\end{figure}

\paragraph The model claims visual learners can learn best when consuming
content visually, through pictures, visual aids such as overhead slides,
blackboard visuals graphs, or diagrams. Auditory learners benefit from listening
to lectures, oral discussions, or tapes. Learners of the reading/writing variety
benefit most from committing their ideas to paper, note taking, and relevant
reading material.  Finally, kinesthetic learners prefer to learn via physical
activities, such as touching and experimenting.

\section{Experiential Learning Model}

\paragraph One of the oldest models of learning styles, proposed in 1984, is
American educational theorist David Kolb's Experiential Learning Model
\cite{kolb1984experiential}. This model outlines two related approaches for
\textit{grasping} experience: Concrete Observation and Abstract
Conceptualization, as well as two other related approaches for
\textit{transforming} experience: Reflective Observation and Active
Experimentation. The theory describes a cyclical model of learning that provides
a feedback loop where "knowledge is created through the transformation of
experience" \cite{kolb1984experiential}.

\begin{figure}[H]
    \centering
    \includegraphics[width=5in]{figures_tables/kolb}
    \caption{The Experiential Learning model}
\end{figure}

\paragraph Using these four types of experiential learning, Kolb devised a
learning style inventory which can be used to identify four unique methods of
learning based on which experiential approach is dominant. Each style of
learning represents a choice. As it is impossible to simultaneously play an
instrument (Concrete Experience) and read an instructional book on guitar playing
(Abstract Conceptualization), it is necessary to resolve the conflict by making
a choice between the two methods. As one begins developing a preference as to
which method of learning one is best suited, this accumulated characteristic
becomes what is known as an Experiential Learning style.

\begin{singlespacing}
    \begin{table}[H]
        \hspace{-0.5in}
        \begin{tabular}{|c|p{2in}|p{2.5in}|}
            \hline
            \textbf{Learning Style} & \textbf{Learning Characteristic} & \textbf{Description} \\ \hline
            \textbf{Converger} & Abstract conceptualization \newline + active experimentation &
            - strong in practical application of ideas \newline
            - can focus on deductive reasoning on specific problems \newline
            - unemotional \\ \hline
            \textbf{Diverger} & Concrete experience \newline + reflective observation &
            - strong imaginative ability \newline
            - interested in people \newline
            - good at generating ideas and seeing things from different perspectives \\ \hline
            \textbf{Assimilator} & Abstract conceptualization \newline + reflective observation &
            - strong ability to create theoretical models \newline
            - concerned with abstract concepts rather than people \newline
            - excels in inductive reasoning \\ \hline
            \textbf{Accommodator} & Concrete experience \newline + active experimentation &
            - greatest strength is doing things \newline
            - solves problems intuitively \newline
            - performs well when required to react quickly \\ \hline
        \end{tabular}
        \caption{Kolb's learning style inventory}
    \end{table}
\end{singlespacing}

\pagebreak

\section{Honey and Mumford Model}

\paragraph The Honey and Mumford model of learning styles is derived directly
from Kolb's Experiential Learning model, with some changes made to address
insufficiencies when dealing with the managerial experiences of decision making
and problem solving.

\begin{figure}[H]
    \centering
    \includegraphics[width=5in]{figures_tables/honey_mumford}
    \caption{The Honey and Mumford model}
\end{figure}

\paragraph Honey and Mumford used the learning style inventory of the
Experiential Learning model and produced the Learning Styles Questionnaire: a
simple agree/disagree questionnaire designed to help uncover learning
preferences in individuals who have may not have ever given any consideration as
to how they learn best.

\chapter*{Critique of Learning Styles}
\addcontentsline{toc}{chapter}{Critique of Learning Styles}
\setcounter{chapter}{3}

\paragraph Although the theory of learning styles claims to be able to provide a
more effective and efficient educational model, there is a significant lack of
rigorous scientific research to support these claims. In order to properly
demonstrate the benefits of a learning style model, it is necessary to first
identify the individual learning style of a student according to the model which
is under scrutiny and then randomly assign the student to any of the learning methods
included in the model. The student's performance on a variety of tasks would
then be thoroughly tested before and after instruction in order to determine
what effect the application of a particular learning style had on the student.

\paragraph However, researchers find that studies designed to test the benefits
of learning styles have failed to use the type of randomized research strategies
that would ensure scientifically credible results. An article published for the
Association of Psychological Science states:

\begin{quote}
    We found virtually no evidence for learning styles$\dots$ very few studies
    have even used an experimental methodology capable of testing the validity
    of learning styles applied to education. Moreover, of those that did use an
    appropriate method, several found results that flatly contradict the popular
    hypothesis" \cite{pashler2008learning}.
\end{quote}

\noindent Even though the authors of the article did find one experimental study which
was conducted in such a way as to potentially provide strong evidence for the
learning style hypothesis, that study was not without flaws. The study
tested 324 ``gifted and talented" high-school students using the Sternberg
Triarchic Abilities Test \cite{sternberg1993sternberg}, which rates each
student's analytical, creative and practical abilities. Based on the test
results, a subset of students were assigned to groups where they received
instruction tailored to their specific area of strength. The study found that
``after the data were screened for deviant scores, matched subjects reliably
outscored mismatched subjects on two of the three kinds of assessments."
\cite{sternberg1996identification}.

\paragraph Unfortunately, the study suffers from some serious methodological
issues. The positive correlation between the method of instruction and higher
test results was only achieved after some significant post-processing: removing
the deviant scores mentioned above. As well, the final test scores used
to determine a subject's improvement were never reported.

\paragraph In other cases, research efforts could find no positive relationship
between the application of a learning style method and actual enhanced learning.
Studies such as Massa and Mayer \cite{massa2006testing} and Constantinidou and
Baker \cite{constantinidou2002stimulus}, find no reason to believe the practical
application of learning styles has a positive impact on learning.

\chapter*{Conclusions and Recommendations}
\addcontentsline{toc}{chapter}{Conclusions and Recommendations}
\setcounter{chapter}{4}
\setcounter{section}{0}

\section{Conclusions} %% at least 3 conclusions

\paragraph Learning styles have been a popular idea throughout modern education
theory and, while much effort has been put into producing a large body of
research regarding the theory, there has yet to be any scientifically credible
experimentation to affirm the claims. Too few of the studies conducted are
rigorous or methodological enough to hold up to scientific scrutiny. The few
that can often produce results that do not support the learning styles
hypothesis. As such, there is currently little scientific credibility lent to
the idea that the use of learning styles in a modern education setting provides
any kind of beneficial effect.

\paragraph While these deep rooted issues with learning styles are certainly
troubling, there are also flaws that can arise during the implementation of a
learning style method. Most commonly, learning style questionnaires are used to
measure an individual's learning style preference. However this is inherently
biased. Self-reports such as these very rarely produce a true sampling of
learning behaviour, but instead reflect the learner's own impression of how
they learn, or an similarly influenced impression which may be affected by what
the subject thinks the conductor of the questionnaire wants to hear. Methods
such as these learning style questionnaires or other such self evaluations
cannot be used as a reliable instrument with which to conduct research.

\section{Recommendations} %% at least 2 recommendations

\paragraph The author advises any readers interested in education theory and the
psychology behind human learning to think critically about the theory of
learning styles. Before blindly incorporating learning style based methods into
a curriculum or educational program, consider first the lack of credible results
regarding the proposed benefits of the theory. Many studies conclude that the
positive effect of a learning style based curriculum is negligible, and thus the
resources spent investing in the program are wasted.

\paragraph While the application of learning styles cannot be shown to produce a
tangible improvement in the learning process, what can be taken from the theory
is that the process of human learning exists as an incredibly varied spectrum of
possibilities. Instead of narrowing one's focus on one percieved "special"
section of this spectrum, it is wise to instead attempt to incorporate a
multitude of methods in order to provide a rich, varied curriculum that will engage
a wide audience of learners. The author recommends that one take into account
the important characteristics of learning that each learning style method has
classified and tailor an educational program to target these characteristics in
concert rather than isolation. Instead of hoping to find gains on an individual
level, this approach will allow you to create a learning process that provides
benefit to a deeper range of learners.

\bibliography{references}

\end{document}
