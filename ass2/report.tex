\documentclass[12pt]{article}
\usepackage[margin=1.5in]{geometry}
\usepackage{graphicx}
\usepackage{float}
\usepackage{setspace}
\doublespacing

\bibliographystyle{plain}

\makeatletter
\renewcommand*\l@section[2]{
    \ifnum \c@tocdepth >\m@ne
    \addpenalty{-\@highpenalty}
    \addvspace{1.0em \@plus\p@}
    \setlength\@tempdima{1.5em}
    \begingroup
    \parindent \z@ \rightskip \@pnumwidth
    \parfillskip -\@pnumwidth
    \leavevmode \bfseries
    \advance\leftskip\@tempdima
    \hskip -\leftskip
    #1\nobreak
    \xleaders\hbox{$\m@th
    \mkern \@dotsep mu\hbox{.}\mkern \@dotsep mu$}
    \hfil\nobreak\hb@xt@\@pnumwidth{\hss #2}\par
    \penalty\@highpenalty
    \endgroup
\fi}
\makeatother

\begin{document}

\tableofcontents

\addcontentsline{toc}{section}{List of Figures}
\listoffigures

\addcontentsline{toc}{section}{List of Tables}
\listoftables

\pagebreak

\addcontentsline{toc}{section}{Summary} %% 1 page

\pagebreak

\section{Introduction} %% 1 page

\paragraph The theory of learning styles has been around in some form or
another for more than 40 years. Originated in the 1970's, the concept of
learning styles has had a widespread influence on modern education although the
credibility of the theory has come under serious question. This report will
first cover the most popular flavours of the learning styles such as Neil
Fleming's \textit{VARK} \cite{fleming2001teaching}, David Kolb's
\textit{Experiential Learning} \cite{kolb1984experiential}, and Peter Honey and
Alan Mumford's \textit{Learning Styles Questionnare} \cite{}. The core ideas of
learning styles will be critically examined, including the proposed benefits
such as increaded content retention, increased ability to understand new
material, and enhanced scores on testable material.

\paragraph After presenting the most prevalent models of learning styles, this
report will then go on to discuss the modern critique of the theory. Although
the theory has existed for a substantial period, there is an alarming lack of
scientifically rigorous studies to confirm the claims touted by proponents of
the theory. Though there have been attempts to ground the theory in factual,
reproducable experimentation, none have provided a satisfyingly conclusive
answer to the question of whether the application of learning style theories in
an educational setting has real, tangible benefit to the learning process. This
report will draw its own conclusions as to the suitability of learning styles
as an educational tool, and provide recommendations for any readers interested
in employing a model of learning styles in an educational settingover 40 years

%% BODY 4-6 pages, plus tables/figures

\section{An Explanation of Learning Styles}

\paragraph Although many distinct models of learning styles exist, they share a
core theme.  Learning styles are broadly described as ``cognitive, affective,
and physiological traits that are relatively stable indicators of how learners
perceive, interact with, and respond to the learning environment''
\cite{keefe1990developing}. This suggests that an individual will be able to
learn at an increased rate, or with higher content retention, if they are
taught in a way that aligns with an innate personal preference for consuming
information.

\paragraph One well known model is Neil Fleming's \textit{Visual, Auditory,
Reading/Writing, Kinesthetic} or VARK model \cite{fleming2001teaching}.
Fleming himself describes VARK as "a starting place for a conversation among
teachers and learners about learning" \cite{fleming2006learning}.  This claims
visual learners can learn best when consuming content visually, through
pictures, visual aids such as overhead slides, graphs, or diagrams. Auditory
learners benefit from listening to lectures, oral discussions, or tapes.
Finally kinesthetic learners prefer to learn via physically doing, touching,
and experimenting.

\begin{figure}[H]
    \centering
    %\includegraphics[width=5in]{figures_tables/vark}
    \caption{The VARK model}
\end{figure}

\paragraph One of the oldest models of learning styles is Kolb's
\textit{Experiential Learning} \cite{kolb1984experiential}, proposed in 1984.
This model outlines two related approaches for \textit{grasping} information:
Concrete Observation and Abstract Conceptualization as well as two other
related approached for \textit{transforming} experience: Reflective Observation
and Active Experimentation. According to the model, an ideal learning process
will encorporate all these experiential opportunities. However

\begin{singlespacing}
    \begin{table}[H]
        \begin{tabular}{|l|p{5cm}|p{6cm}|}
            \hline
            \textbf{Learning Style} & \textbf{Learning Characteristic} & \textbf{Description} \\ \hline
            \textbf{Converger} & Abstract conceptualization + active experimentation &
            - strong in practical application of ideas \newline
            - unemotional \newline
            - can focus on hypo-deductive reasoning on specific problems \\ \hline
            \textbf{Diverger} & Concrete experience + reflective observation &
            - strong imaginative ability \newline
            - interested in people \newline
            - good at generating ideas and seeing things from different perspectives \\ \hline
            \textbf{Assimilator} & Abstract conceptualization + reflective observation &
            - strong ability to create theoretical models \newline
            - concerned with abstract concepts rather than people \newline
            - excels in inductive reasoning \\ \hline
            \textbf{Accommodator} & Concrete experience + active experimentation &
            - greatest strength is doing things \newline
            - solves problems intuitively \newline
            - performs well when required to react to immediate cirumstances \\ \hline
        \end{tabular}
        \caption{An inventory of Experiential Learning Styles}
    \end{table}
\end{singlespacing}

\begin{figure}[H]
    \centering
    %\includegraphics[width=5in]{figures_tables/kolb}
    \caption{The Experiential Learning model}
\end{figure}

\section{Critique of Learning Styles}

\paragraph Although the theory of learning styles claims to be able to provide
a more effective and efficient educational model, there has been a significant
lack of rigorous scientific research to support these claims. In order to
properly demonstrate the benefits of a learning style model, it is nessecary to
first identify the individual learning style of a student according to the
model which is under scrutiny and then randomly assign them to any of the
learning methods included in the model. The student's performance on a variety
of tasks would then be thoroughly tested before and after instruction in
order to determine if the application of a particular learning style positively
impacted the student.

\section{An Assessment of my Learning Style}

\section{Conclusions} %% at least 3 conclusions

\paragraph As discussed, the theory of learning styles has been largely
discredited and debunked. Although much effort and research has been directed
towards the theory, few are rigorous or conclusive enough to lend any real
credibility to the idea that the use of learning styles in a modern education
setting provides any kind of beneficial effect. The studies, while numerous,
fail to show any strong correlation between the usage of learning styles in an
educational setting, and improved content retention or higher grades on tests.
In many cases, these studies are plagued by confirmation bias. The researcher's
efforts to provide extra material to appeal to different learning styles only
manages to prove that additional content or instruction improves the learning
process, while the benifits cannot be directly attributed to the success of any
Learning Style based theories.

\paragraph While learning styles attempts to neatly characterize the infinitude
of possibilities for human learning into well-defined categories, the emerging
truth is that the learning process is much more subtle and complex. Attempting
to restrict learning to such clearly disjoint separations such as the VARK
model then becomes an exercise in failure.

\section{Recommendations} %% at least 2 recommendations

\paragraph The author advises any readers interested in the education process
and the psychology behind human learning to think critically about the theory
of learning styles Before blindly incorporating Learning Style related
methodology into a curriculum or educational program, consider the lack of
credible results regarding any possible benefits. It is possible that by
restricting yourself to one misguided model of education, you will limit your
ability to efficiently and effectively instruct. Instead, focus on some of the
orthogonal results that have come from research into learning styles. It has
been conclusively proven that providing a wide variety of educational
resources, as well as stimulating, engaging material contribute positively to
material rentention.

\bibliography{references}

\end{document}
