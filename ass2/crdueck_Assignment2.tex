\documentclass[12pt]{report}

\usepackage[margin=1.5in]{geometry}

\bibliographystyle{plain}

\setlength{\parskip}{\baselineskip}

\begin{document}

The theory of learning styles has been around in some form or another for more
than 40 years. Originated in the 1970's, the concept of learning styles has had
a widespread influence on modern education although the credibility of the
theory has come under serious question. This report will first cover the most
popular flavours of the learning styles such as Neil Fleming's \textit{VARK}
\cite{fleming2001teaching}, David Kolb's \textit{Experiential Learning}
\cite{kolb1984experiential}, and Peter Honey and Alan Mumford's \textit{Learning
Styles Questionnare}. The core ideas of learning styles will be critically
examined along with the body of research conducted in the name of learning
styles. This paper will present the modern view on learning styles, namely that
their claims have not quantitively been shown to hold in practice and for what
reasons.

Although many distinct models of learning styles exist, they share a core theme.
Learning styles are broadly described as ``cognitive, affective, and
physiological traits that are relatively stable indicators of how learners
perceive, interact with, and respond to the learning environment''
\cite{keefe1990developing}. This suggests that an individual will be able to
learn at an increased rate, or with higher content retention, if they are taught
in a way that aligns with an innate personal preference for consuming
information.  One well known model is the \textit{Visual, Auditory, Reading,
Kinesthetic} or VARK model. This claims visual learners can learn best when
consuming content visually, through pictures, visual aids such as overhead
slides, graphs, or diagrams. Auditory learners benefit from listening to
lectures, discussions, or tapes. Finally kinesthetic learners prefer to learn
via physically doing, touching, and experimenting.

%% CONCLUSION

As discussed, the theory of learning styles has been largely discredited and
debunked. Although much effort and research has been directed towards the
theory, few are rigorous or conclusive enough to lend any real credibility to
the idea that the use of learning styles in a modern education setting provides
any kind of beneficial effect. The studies, while numerous, fail to show any
strong correlation between the usage of learning styles in an educational
setting, and improved content retention or higher grades on tests. In many
cases, these studies are plagued by confirmation bias. The researcher's efforts
to provide extra material to appeal to different learning styles only manages to
prove that additional content or instruction improves the learning process,
while the benifits cannot be directly attributed to the success of any Learning
Style based theories.

While learning styles attempts to neatly characterize the infinitude of
possibilities for human learning into well-defined categories, the emerging
truth is that the learning process is much more subtle and complex.  Attempting
to restrict learning to such clearly disjoint separations such as the VARK model
then becomes an exercise in failure.

The author advises any readers interested in the education process and the
psychology behind human learning to think critically about the theory of
learning styles Before blindly incorporating Learning Style related methodology
into a curriculum or educational program, consider the lack of credible results
regarding any possible benefits. It is possible that by restricting yourself to
one misguided model of education, you will limit your ability to efficiently and
effectively instruct. Instead, focus on some of the orthogonal results that have
come from research into learning styles.  It has been conclusively proven that
providing a wide variety of educational resources, as well as stimulating,
engaging material contribute positively to material rentention.

\bibliography{refs}

\end{document}
